% This is samplepaper.tex, a sample chapter demonstrating the
% LLNCS macro package for Springer Computer Science proceedings;
% Version 2.20 of 2017/10/04
%
\documentclass[runningheads]{llncs}
%
\usepackage{amsmath}
\usepackage{amssymb}
%\usepackage{amsthm}
\usepackage{mathtools}
%
\usepackage{todonotes}
%
\usepackage{hyperref}
\usepackage{cleveref}
%
\usepackage{graphicx}
% Used for displaying a sample figure. If possible, figure files should
% be included in EPS format.
%
% If you use the hyperref package, please uncomment the following line
% to display URLs in blue roman font according to Springer's eBook style:
\renewcommand\UrlFont{\color{blue}\rmfamily}

\newcommand{\prover}{\mathsf{P}}
\newcommand{\verifier}{\mathsf{V}}

\newcommand{\GG}{\mathcal{G}}
\newcommand{\HH}{\mathcal{H}}

\newcommand{\CC}{\mathcal{C}}

\newcommand{\ZZ}{\mathbb{Z}}

\newcommand{\getsr}{\gets_{\$}}

\newcommand{\minote}[1]{\todo[color=green!30, inline]{\textbf{Michele's note:} #1}}

\begin{document}

\title{Proposal: $\Sigma$-protocols%
\thanks{This work has partially been funded by the European Union's Horizon 2020 framework programme under grant agreement no. 830929 (CyberSec4Europe).}}
%
\titlerunning{Proposal: $\Sigma$-protocols}

\author{Stephan Krenn\and
        Michele Orr\`u}

\authorrunning{S. Krenn and M. Orr\`u}

\institute{AIT Austrian Institute of Technology, Vienna, Austria \and
           University of California, Berkeley, United States}
%
\maketitle              % typeset the header of the contribution
%
\begin{abstract}
The abstract should briefly summarize the contents of the paper in
150--250 words.

\keywords{First keyword  \and Second keyword \and Another keyword.}
\end{abstract}

\section{Introduction}
\subsection{Related Work}

\section{Background and Motivation}

\section{Notation and Terminology}

\section{Constructions for $\Sigma$-Protocols}
\subsection{Basic $\Sigma$-Protocols in Prime-Order Groups}
A basic $\Sigma$-protocol to prove knowledge of a preimage $x$ of $y$ under a homomorphism $\varphi:\GG\to\HH$ consists of three messages being exchanged:

\begin{enumerate}
  \item
    In a first step, the prover $\prover$ chooses a random element $r\getsr\GG$ and computes $t\gets\varphi(r)$.
	It sends $t$ to the verifier $\verifier$.
  \item
    Next, $\verifier$ chooses $c\getsr\CC$ and sends $c$ back to the prover.
  \item
    The prover checks that $c\in\CC$ and aborts if this is not the case.
	It computes its response as $s\gets r+cx$, which it sends to the verifier.
  \item
    The verifier checks that $s\in\GG$ and $t\in\HH$, and outputs $0$ if this is not the case.
	It then checks whether $t + cy = \varphi(s)$, and outputs $1$ if this is the case; otherwise, $\verifier$ outputs $0$.
\end{enumerate}

\begin{theorem}
  If $\HH$ is a cyclic group of prime order $q$, $\varphi:\GG\to\HH$ is a group homomorphism, and $\CC=\ZZ_m$ for some $m\leq q$, then the above $\Sigma$-protocol is an honest-verifier zero-knowledge proof of knowledge for the relation $y=\varphi(x)$.
\end{theorem}

\subsection{Composition of $\Sigma$-Protocols}
\subsubsection{AND Composition}

\subsubsection{OR Composition}

\subsection{Achieving Non-Interactivity - The Fiat-Shamir Transform}


\section{Security Considerations}

\section{Implementation}

\bibliographystyle{splncs04}
\bibliography{cryptobib/abbrev3,cryptobib/crypto}
%
\end{document}
